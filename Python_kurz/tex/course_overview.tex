\documentclass[12pt]{article}
\setlength{\oddsidemargin}{27mm}
\setlength{\evensidemargin}{27mm}
\setlength{\hoffset}{-1in}

\setlength{\topmargin}{27mm}
\setlength{\voffset}{-1in}
\setlength{\headheight}{0pt}
\setlength{\headsep}{0pt}

\setlength{\textheight}{235mm}
\setlength{\textwidth}{155mm}

%\pagestyle{empty}
\pagestyle{plain}
\usepackage{graphicx}
\usepackage{amsmath}
\usepackage[slovak]{babel}
\usepackage{hyperref}
\hypersetup{
    colorlinks,
    citecolor=black,
    filecolor=black,
    linkcolor=black,
    urlcolor=black
}

\begin{document}
\baselineskip 15pt

\begin{center}
\textbf{\Large Základy dátovej analýzy v Pythone}

\vspace{1.5cc}
{ \sc Martin Vozár}

\end{center}

\vspace{0.3 cm}


\section*{Cieľ kurzu}
Očakávaným výsledkom tohoto kurzu je učastníka prakticky previesť a doviesť
k vlastnej realizácií pri základnej práci spracovania a vizualizácie dát.
\newline \newline
Kurz bude pozostávať zo série praktických cvičení spojených s výkladom
teórie podľa rozsahu kurzu.
\newline \newline
Počas cvičenia účastník produkuje vlastný kód v jazyku Python,
prípadne v prostredí Jupyter Notebook. Načítanie a spracovanie bude
realizované za pomoci niektorých zaužívaných knižníc - numpy, pandas,
scipy, matplotlib, v prípade pokročilejších metód scikit-learn,
tensorflow.
\newline \newline
Medzi aplikované metódy budú v základnej variante kurzu patriť 
lineárna a parametrická regresia. Vo variante pre pokročilejších
účastníkov bude navyše aplikácie klasifikačného algoritmu 
rozhodovací strom. Možnosťou pri vyššej vstupnej znalosti účastníkov je
implemetovať neurónové siete, či iné pokročilejšie metódy.

\section*{Očakávané vstupné znalosti}
V základnej variante kurzu účastník nepotrebuje žiadne predchádzajúce
znalosti, či skúsenosti. V tejto variante je úvodná časť kurzu obohatená
o prechod základmi jazyka Python vrátane inštalácie prostredia
postačujúceho (nie len) pre absolvovanie kurzu.
\newline \newline
Vo všeobecnosti je výhodou, vo variante pre pokročilých prerekvizitou
znalosť základou jazyka Python, práca s knižnicami, schopnosť čítania
dokumentácie knižníc pri samostanom adresovaní problémov.

\section*{\textit{Disclaimer}}
\textit{Tento dokument je pracovnou verziou návrhu kurzu. Jeho finalizácia
        je predmetom ďalšej činnosti.
}

\newpage
\tableofcontents

\newpage
\section*{Úvod}
\label{sec:intro}
\addcontentsline{toc}{section}{\nameref{sec:intro}}

\section{Inštalácia a zoznámenie sa s prostredím}
Cieľom tejto časti je nainštalovať u účastníkov spoľahlivo funkčné
prostredie pre ďalšiu prácu. Súčasťou tohoto procesu je nainštalovanie
príslušného prostredia, pričom účastníci su prevedení jednou z viacerých
variánt tejto inštalácie. V procese tejto inštalácie môžu byť oboznámení
s alternatívami k jednotlivým prvkom.

Účastníci kurzu sú prevedený inštaláciou interpretera
Python. Následne je predstavený package manager pip, spomenutý package
manager Anaconda. Ďalej práca s package managerom pip na inštaláciu
knižníc. Rýchly priebeh inštaláciou VSCode, Jupyter Lab, ako vhodných
prostredí.

Začína sa úvodným programom "Hello World!". Nasleduje import
knižnice numpy a výpis základných typov poľa ako np.array, np.arrange,
np.linspace. Pokračuje zavádzanie natívnej funkcie v jazyku Python
a práca s knižnicou numpy v tejto súvislosti.

Nasleduje import knižnice pandas a základná práca s ňou. Primárne
pôjde o zoznámeni sa s obejaktami pd.DataFrame, pd.Series. Prevedenie
prechodu z pd.Series do np.array.
\newline \newline
Import knižnice matplotlib, resp. matplotlib.pyplot a vizualizácia
individuálnych sérií, vizualizácia viacerých sérií. Zoznámenie sa
s niektorými možnosťami grafickej knižnice v zámere produkovať
čitateľné a prehľadné grafy.

\section{Práca s datasetom Iris}
\subsection{Načítanie dát zo súboru}
Prevedenie načítania dát zo súboru do dátových typov známych z predchádzajúceho
celku. Následuje zobrazenie dát v dvojrozmernom priestore s kategóriou
znázornenou farbou datapointu na grafe. Vysvetlenie rozdielu medzi spojitou
a kategorickou veličinou. Nahrádzanie kategorickej veličiny celočíselným indexom.

\subsection{Regresia}
Import knižnice scipy, resp. funkcie scipy.optimize.curve\_fit na realizáciu
optimalizácie.
Na závislosti jednotlivých veličín je aplikovaná lineárna regresia. Vysvetlenie
princípu metódy najmenších štvorcov.
Aplikácia parametrickej regresie. Vysvetlenie princípu optimalizácie funkcie
viacerých voľných paramtrov.

\end{document}


%  This note\footnote{Modified from ``WIC Symposium 2009 - Paper Instructions'' by
% F. Willems and T. Tjalkens} provides instructions for the preparation and
% submission of the final versions of the accepted papers for the
% 38$^{\textrm{th}}$ WIC\footnote{Werkgemeenschap voor Informatie- en Communicatietheorie}
% Symposium on Information Theory in the Benelux (SITB), and 7$^{\textrm{th}}$ joint
% WIC/IEEE SP Symposium on Information Theory and Signal Processing in the Benelux to be
% held in Enschede, the Netherlands, May 31--June 1, 2018   (see
% \verb+http://www.utwente.nl/sitb2018+).

